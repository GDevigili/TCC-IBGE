% -----------------------------------
% -----------------------------------
% abnTeX2: Normas ABNT NBR 14724:2011 + sugestões FGV/EMAp. 

% Autor: Lauro César Araujo
% Adaptações EMAp: Lucas Machado Moschen 
% Copyright 2012-2018 by abnTeX2 group at http://www.abntex.net.br/ 

%% This work may be distributed and/or modified under the
%% conditions of the LaTeX Project Public License, either version 1.3
%% of this license or (at your option) any later version.
%% The latest version of this license is in
%%   http://www.latex-project.org/lppl.txt
%% and version 1.3 or later is part of all distributions of LaTeX
%% version 2005/12/01 or later.
% ----------------------------------
% ----------------------------------
\documentclass[
	% -- opções da classe memoir --
	12pt,				% tamanho da fonte
	%openright,			% capítulos começam em página ímpar (insere página vazia caso preciso)
	oneside,			% para impressão em recto e verso. Oposto a oneside
	a4paper,			% tamanho do papel. 
	% -- opções da classe abntex2 --
	%chapter=TITLE,		% títulos de capítulos convertidos em letras maiúsculas
	%section=TITLE,		% títulos de seções convertidos em letras maiúsculas
	%subsection=TITLE,	% títulos de subseções convertidos em letras maiúsculas
	%subsubsection=TITLE,% títulos de subsubseções convertidos em letras maiúsculas
	% -- opções do pacote babel --
	english,			% idioma para inglês
	brazil				% idioma para português
	]{abntex2}

%------------------------------------------------
%-------------- Pacotes necessários -------------
%------------------------------------------------

% Escrita 
\usepackage[T1]{fontenc}
\usepackage[utf8]{inputenc}
\usepackage{lmodern}
\usepackage{microtype} % para melhorias de justificação
\usepackage{indentfirst}

\renewcommand{\ABNTEXchapterfont}{\fontfamily{ptm}\fontseries{b}\selectfont}

% Gráficos 
\usepackage{color}
\usepackage{caption}
\usepackage{subcaption}
\usepackage{graphicx}
\graphicspath{{../../images/}}
\usepackage{xcolor}

% Matemáticos 
\usepackage{amsthm, amssymb, amsmath, mathtools}

% Outros 
\usepackage{lipsum}
\usepackage{listings}
\usepackage{minted}

\usepackage{xurl}

% Citações 
%\usepackage[brazilian,hyperpageref]{backref}
%\usepackage[alf]{abntex2cite}	% Citações padrão ABNT
\usepackage[style=abnt]{biblatex}
\addbibresource{biblio.bib}  

% \renewcommand{\backrefpagesname}{Citado na(s) página(s):~}
% % Texto padrão antes do número das páginas
% \renewcommand{\backref}{}
% % Define os textos da citação
% \renewcommand*{\backrefalt}[4]{
% 	\ifcase #1 %
% 		Nenhuma citação no texto.%
% 	\or
% 		Citado na página #2.%
% 	\else
% 		Citado #1 vezes nas páginas #2.%
% 	\fi}%
% ---


% Configurações do listings


\definecolor{lightgray}{rgb}{.9,.9,.9}
\definecolor{darkgray}{rgb}{.4,.4,.4}
\definecolor{purple}{rgb}{0.65, 0.12, 0.82}

\lstdefinelanguage{Python}{
  keywords={typeof, True, False, try, except, return, None, catch, var, if, in, while, do, else, case, break, elif, for, def, class, import},
  keywordstyle=\color{blue}\bfseries,
  ndkeywords={class, export, bool, raise, import, self, def},
  ndkeywordstyle=\color{darkgray}\bfseries,
  identifierstyle=\color{black},
  sensitive=false,
  comment=[l]{\#},
  morecomment=[s]{"""}{"""},
  commentstyle=\color{green}\ttfamily,
  stringstyle=\color{red}\ttfamily,
  morestring=[b]',
  morestring=[b]",
  escapeinside={\%*}{*)},
   literate=  {á}{{\'a}}1 {é}{{\'e}}1 {í}{{\'i}}1 {ó}{{\'o}}1 {ú}{{\'u}}1
  {Á}{{\'A}}1 {É}{{\'E}}1 {Í}{{\'I}}1 {Ó}{{\'O}}1 {Ú}{{\'U}}1
  {à}{{\`a}}1 {è}{{\`e}}1 {ì}{{\`i}}1 {ò}{{\`o}}1 {ù}{{\`u}}1
  {À}{{\`A}}1 {È}{{\'E}}1 {Ì}{{\`I}}1 {Ò}{{\`O}}1 {Ù}{{\`U}}1
  {ä}{{\"a}}1 {ë}{{\"e}}1 {ï}{{\"i}}1 {ö}{{\"o}}1 {ü}{{\"u}}1
  {Ä}{{\"A}}1 {Ë}{{\"E}}1 {Ï}{{\"I}}1 {Ö}{{\"O}}1 {Ü}{{\"U}}1
  {â}{{\^a}}1 {ê}{{\^e}}1 {î}{{\^i}}1 {ô}{{\^o}}1 {û}{{\^u}}1
  {Â}{{\^A}}1 {Ê}{{\^E}}1 {Î}{{\^I}}1 {Ô}{{\^O}}1 {Û}{{\^U}}1
  {ç}{{\c c}}1 {Ç}{{\c C}}1 {ø}{{\o}}1 {å}{{\r a}}1 {Å}{{\r A}}1
  {ã}{{\~a}}1 {Ã}{{\~A}}1 {õ}{{\~o}}1 {Õ}{{\~O}}1
}

\lstdefinelanguage{Stata}{
  keywords={quietly, infix, byte, using, clear, format, \%, label, var, define, add, values},
  keywordstyle=\color{blue}\bfseries,
  ndkeywords={class, export, bool, raise, import, self, def},
  ndkeywordstyle=\color{darkgray}\bfseries,
  identifierstyle=\color{black},
  sensitive=false,
  comment=[l]{*},
  %morecomment=[s]{"""}{"""},
  commentstyle=\color{green}\ttfamily,
  stringstyle=\color{red}\ttfamily,
  %morestring=[b]',
  morestring=[b]",
  escapeinside={\%*}{*)},
   literate=  {á}{{\'a}}1 {é}{{\'e}}1 {í}{{\'i}}1 {ó}{{\'o}}1 {ú}{{\'u}}1
  {Á}{{\'A}}1 {É}{{\'E}}1 {Í}{{\'I}}1 {Ó}{{\'O}}1 {Ú}{{\'U}}1
  {à}{{\`a}}1 {è}{{\`e}}1 {ì}{{\`i}}1 {ò}{{\`o}}1 {ù}{{\`u}}1
  {À}{{\`A}}1 {È}{{\'E}}1 {Ì}{{\`I}}1 {Ò}{{\`O}}1 {Ù}{{\`U}}1
  {ä}{{\"a}}1 {ë}{{\"e}}1 {ï}{{\"i}}1 {ö}{{\"o}}1 {ü}{{\"u}}1
  {Ä}{{\"A}}1 {Ë}{{\"E}}1 {Ï}{{\"I}}1 {Ö}{{\"O}}1 {Ü}{{\"U}}1
  {â}{{\^a}}1 {ê}{{\^e}}1 {î}{{\^i}}1 {ô}{{\^o}}1 {û}{{\^u}}1
  {Â}{{\^A}}1 {Ê}{{\^E}}1 {Î}{{\^I}}1 {Ô}{{\^O}}1 {Û}{{\^U}}1
  {ç}{{\c c}}1 {Ç}{{\c C}}1 {ø}{{\o}}1 {å}{{\r a}}1 {Å}{{\r A}}1
  {ã}{{\~a}}1 {Ã}{{\~A}}1 {õ}{{\~o}}1 {Õ}{{\~O}}1
}

\lstset{
   %backgroundcolor=\color{lightgray},
   extendedchars=true,
   %basicstyle=\footnotesize\ttfamily,
   showstringspaces=false,
   showspaces=false,
   %numbers=left,
   %numberstyle=\footnotesize,
   numbersep=9pt,
   tabsize=2,
   breaklines=true,
   showtabs=false,
   captionpos=b,
   escapeinside={\%*}{*)},
   literate=  {á}{{\'a}}1 {é}{{\'e}}1 {í}{{\'i}}1 {ó}{{\'o}}1 {ú}{{\'u}}1
  {Á}{{\'A}}1 {É}{{\'E}}1 {Í}{{\'I}}1 {Ó}{{\'O}}1 {Ú}{{\'U}}1
  {à}{{\`a}}1 {è}{{\`e}}1 {ì}{{\`i}}1 {ò}{{\`o}}1 {ù}{{\`u}}1
  {À}{{\`A}}1 {È}{{\'E}}1 {Ì}{{\`I}}1 {Ò}{{\`O}}1 {Ù}{{\`U}}1
  {ä}{{\"a}}1 {ë}{{\"e}}1 {ï}{{\"i}}1 {ö}{{\"o}}1 {ü}{{\"u}}1
  {Ä}{{\"A}}1 {Ë}{{\"E}}1 {Ï}{{\"I}}1 {Ö}{{\"O}}1 {Ü}{{\"U}}1
  {â}{{\^a}}1 {ê}{{\^e}}1 {î}{{\^i}}1 {ô}{{\^o}}1 {û}{{\^u}}1
  {Â}{{\^A}}1 {Ê}{{\^E}}1 {Î}{{\^I}}1 {Ô}{{\^O}}1 {Û}{{\^U}}1
  {œ}{{\oe}}1 {Œ}{{\OE}}1 {æ}{{\ae}}1 {Æ}{{\AE}}1 {ß}{{\ss}}1
  {ç}{{\c c}}1 {Ç}{{\c C}}1 {ø}{{\o}}1 {å}{{\r a}}1 {Å}{{\r A}}1
  {€}{{\EUR}}1 {£}{{\pounds}}1 {ã}{{\~a}}1,
   frame=lines,
   numberbychapter=false
}

\lstloadlanguages{Python, Stata}







\input{files/capa_folha_rosto.tex}

\titulo{Criando um repositório de dados públicos: estudo de caso com o Censo Demográfico do IBGE}
\autor{Gianlucca Devigili}
\local{Rio de Janeiro}
\data{2023}
\instituicao{%
  Fundação Getulio Vargas \\
  \par
  Escola de Matemática Aplicada
}
\tipotrabalho{Trabalho de Conclusão de Curso}

\preambulo{Trabalho de conclusão de curso apresentada para a Escola de
Matemática Aplicada (FGV/EMAp) como requisito para o grau de bacharel em
Ciência de Dados e Inteligência Artificial. \\ \\ Área de estudo: ciência de dados.}

\orientador{Júlio César Chaves}

% Se o seu texto tem subtítulo. 
% Se não tiver, altere o arquivo capa_folha_rosto_tex
%\subtitulo{Este é o subtítulo do meu TCC}

%---------------------------------------------
%-------------------- PDF --------------------
%---------------------------------------------

% alterando o aspecto da cor azul
\definecolor{blue}{RGB}{41,5,195}

% informações do PDF
\makeatletter
\hypersetup{
     	%pagebackref=true,
		pdftitle={\@title}, 
		pdfauthor={\@author},
    	pdfsubject={\imprimirpreambulo},
	    pdfcreator={LaTeX with abnTeX2},
		pdfkeywords={abnt}{latex}{abntex}{abntex2}{trabalho acadêmico}, 
		colorlinks=true,       		% false: boxed links; true: colored links
    	linkcolor=blue,          	% color of internal links
    	citecolor=blue,        		% color of links to bibliography
    	filecolor=magenta,      		% color of file links
		urlcolor=blue,
		bookmarksdepth=4
}
\makeatother

% Posiciona figuras e tabelas no topo da página quando adicionadas sozinhas
% em um página em branco. Ver https://github.com/abntex/abntex2/issues/170
\makeatletter
\setlength{\@fptop}{5pt} % Set distance from top of page to first float
\makeatother

%---------------------------------------
%--------- Mais configurações-----------
%---------------------------------------

% Possibilita criação de Quadros e Lista de quadros.
% Ver https://github.com/abntex/abntex2/issues/176
\newcommand{\quadroname}{Quadro}
\newcommand{\listofquadrosname}{Lista de quadros}

% \newfloat[chapter]{quadro}{loq}{\quadroname}
\newlistof{listofquadros}{loq}{\listofquadrosname}
\newlistentry{quadro}{loq}{0}

% configurações para atender às regras da ABNT
\setfloatadjustment{quadro}{\centering}
\counterwithout{quadro}{chapter}
\renewcommand{\cftquadroname}{\quadroname\space} 
\renewcommand*{\cftquadroaftersnum}{\hfill--\hfill}

\setfloatlocations{quadro}{hbtp} % Ver https://github.com/abntex/abntex2/issues/176

%-----------------------------------------------------
%--------------------- Margens -----------------------
%-----------------------------------------------------

\setlrmarginsandblock{3cm}{2cm}{*}
\setulmarginsandblock{3cm}{2cm}{*}
\checkandfixthelayout

%-----------------------------------------------------
%------ Espaçamentos entre linhas e parágrafos -------
%-----------------------------------------------------

% O tamanho do parágrafo é dado por:
\setlength{\parindent}{1.3cm}

% Controle do espaçamento entre um parágrafo e outro:
\setlength{\parskip}{0.2cm}  % tente também \onelineskip

% compila o índice
\makeindex

%------------------------------------------------------
%----------- Personal Definitions ---------------------
%------------------------------------------------------

\input{files/definitions.tex}

%-------------------------------------------------
%----------------- Document ----------------------
%-------------------------------------------------

\begin{document}

\newcounter{num}
% if num != 1, do not print the pre textual 
\setcounter{num}{1}

\selectlanguage{brazil}
\frenchspacing 

%----------------------------------------------
%--------------- Pré-textuais -----------------
%----------------------------------------------
%\pretextual

\imprimircapa

\ifnum\value{num}=1
{\imprimirfolhaderosto*

\input{files/ficha_catalografica.tex}

%\input{files/errata.tex} %EDITAR?

\begin{folhadeaprovacao}

    \begin{center}
      {\ABNTEXchapterfont\large\MakeUppercase{\imprimirautor}}
  
      \vspace*{\fill}\vspace*{\fill}
      \begin{center}
        \ABNTEXchapterfont\bfseries\large\MakeUppercase{\imprimirtitulo}\normalfont\MakeUppercase{:
        \imprimirsubtitulo}	
      \end{center}
      \vspace*{\fill}
      
      \hfill
      \begin{minipage}{.7\textwidth}
          \imprimirpreambulo \\ \\
          E aprovado em dd/mm/yyyy \\ % EDITAR
          Pela comissão organizadora
      \end{minipage}%
      \vspace*{\fill}
     \end{center}
  
     \assinatura{\imprimirorientador \\ Fundação Getúlio Vargas} 
     \assinatura{Convidado 1 \\ Instituição 1} % EDITAR
     \assinatura{Convidado 2 \\ Instituição 2} % EDITAR
     %\assinatura{\textbf{Professor} \\ Convidado 3}
     %\assinatura{\textbf{Professor} \\ Convidado 4}
\end{folhadeaprovacao}

% \begin{folhadeaprovacao}
% \includepdf{folhadeaprovacao_final.pdf}
% \end{folhadeaprovacao}

\begin{dedicatoria}
    \vspace*{\fill}
    %\noindent
    \hfill
    \begin{minipage}{.6\textwidth}
     Dedico esta dissertação aos meus pais, Ricardo e Karina, sem vocês eu não teria chegado até aqui. % EDITAR
    \end{minipage}
\end{dedicatoria}
 
\begin{agradecimentos}
    Aos meus pais, Ricardo e Karina, por todo o apoio e incentivo desde criança, estando presentes em cada uma das minhas conquistas.
    Agradeço também meu orientador, professor Júlio, pelo auxílio e pelos conselhos ao longo de minha jornada acadêmica e profissional, tanto na iniciação científica quanto no desenvolvimento do presente trabalho.
    Ao CDMC, por ter me proporcionado a possibilidade de estudar na FGV.
    Por fim, agradeço todos os amigos e professores que de alguma forma fizeram parte de minha vida.
\end{agradecimentos}

\begin{epigrafe}
\vspace*{\fill}

\begin{flushright}
    \hspace{7.5cm}
    \textit{
        ``So once you do know what the question actually is, you'll know what the answer means.''} \\
        \textit{Douglas Adams} %EDITAR
\end{flushright}
\end{epigrafe} %EDITAR

\setlength{\absparsep}{18pt} 
\begin{resumo}[Resumo]
 %Segundo a  o resumo deve ressaltar o
 %objetivo, o método, os resultados e as %conclusões do documento. A ordem e a extensão
 %destes itens dependem do tipo de resumo (informativo ou indicativo) e do
 %tratamento que cada item recebe no documento original. O resumo deve ser
 %precedido da referência do documento, com exceção do resumo inserido no
 %próprio documento. (\ldots) As palavras-chave devem figurar logo abaixo do
 %resumo, antecedidas da expressão Palavras-chave:, separadas entre si por
 %ponto e finalizadas também por ponto. Deve ser redigido na terceira
 %pessoa do singular e quanto a sua extensão, o resumo deve ter de 150 a 500
 %palavras.

O presente trabalho objetiva a criação de um repositório de dados públicos, disponibilizando \textit{datasets} coletados através de duas diferentes fontes de dados do Instituto Brasileiro de Geografia e Estatística  (IBGE): a API do serviço de dados e os microdados dos Censos Demográficos. Utilizando a API, foram coletados dados referentes às localidades, indicadores socioeconômicos de diversos países, os chamados agregados e seus metadados. O último grupo consiste da consolidação de respostas das pesquisas promovidas pelo instituto de acordo com diversos critérios, variáveis as quais são armazenadas em arquivos de microdados. Estes arquivos contém de forma anonimizada as respostas individuais de cada entrevistado durante a pesquisa, permitindo análises mais detalhadas do que os agregados. Durante o trabalho, foram tratados de dados do censo demográfico brasileiro, contudo os códigos desenvolvidos são genéricos e podem ser utilizados em outras pesquisas do IBGE desde que sigam o mesmo formato  possuam os mesmos dados. Também foi realizado um estudo de caso, no qual foi exemplificado o uso de ambos os tipos de dados e comparando eles através de visualizações.

Palavras-chave: Censo Demográfico. IBGE. Microdados. Dados públicos.

\end{resumo}

\begin{resumo}[Abstract]
 \begin{otherlanguage*}{english}

This work aims to create a public data repository, making available datasets collected from two different sources from Brazilian Institute of Geography and Statistics (IBGE): the data service API and the microdata from the brazilian demographic census. Using the API, data related to locations, socioeconomic indicators from various countries, the so-called aggregations, and their metadata were collected. The last group consists of the consolidation of many variables from surveys made by the institute according to diverse criteria, whose variables are stored in text files called microdata files. These files anonymize individual answers from each of the intervewee that took part of the census and they allow their user to make more detailed data analysis than they could if using the aggregates. Throughout the project, Brazilian demographic census data were processed and treated; however, the codes are generic and can be used in other IBGE surveys as long as they follow the same format and have the same files. Also, there were made data visualizations in order to exemplify and compare both datasets.


 \end{otherlanguage*}

 Keywords: Brazilian demographic census. IBGE. microdata. Public data.
\end{resumo}

\pdfbookmark[0]{\listfigurename}{lof}
\listoffigures* %EDITAR
\cleardoublepage

% \pdfbookmark[0]{\listofquadrosname}{loq}
% \listofquadros*
% \cleardoublepage

\pdfbookmark[0]{\listtablename}{lot}
\listoftables* %EDITAR
\cleardoublepage

\begin{siglas}
    \item[API] \textit{Application Programming Interface} (Interface de programação de aplicações)
    \item[ASCII] \textit{American Standard Code for Information Interchange} (Código Padrão Americano para o Intercâmbio de Informação)
    \item[CSV] \textit{Comma separated values}
    \item[HTTP] \textit{Hypertext Transfer Protocol} 
    \item[IBGE] Instituto Brasileiro de Geografia e Estatística
    \item[ID] Identificador
    \item[JSON] \textit{JavaScript Object Notation} (Notação de Objeto JavaScript)
    \item[MB] \textit{Megabyte} 
    \item[NoSQL] \textit{Not Only Structured Query Language}
    \item[ODS] \textit{Open Document Spreadsheet}
    \item[ONU] Organização das Nações Unidas
    \item[PNAD] Pesquisa Nacional por Amostra de Domicílios 
    \item[SC] Estado de Santa Catarina
    \item[SSL] \textit{Secure Sockets Layer}
    \item [TSV] \textit{Tab separated values}
    \item[UF] Unidade Federativa ou Unidade da Federação
    \item[URL] \textit{Uniform Resource Locator} (localizador uniforme de recursos)
  \end{siglas}
  
%  \begin{simbolos}
%    \item[$ \Gamma $] Letra grega Gama
%  \end{simbolos}

}\fi

\pdfbookmark[0]{\contentsname}{toc}
\tableofcontents*
\cleardoublepage

% ----------------------------------------------------------
% ELEMENTOS TEXTUAIS
% ----------------------------------------------------------
\textual

\chapter{Introdução}

    %No mundo altamente digitalizado em que vivemos hoje, a alta geração de informação tornou a análise de dados uma ferramenta importantíssima e quase que vital para a humanidade, sendo utilizada por indivíduos e organizações como base para diversas atividades, como responder perguntas e sustentar tomadas de decisão. Empresas utilizam dados para análises de mercado, criação de estratégias e , governos para a criação de novas políticas públicas e destinação de gastos


   A análise de dados cada vez mais se torna uma ferramenta vital e importantíssima para indivíduos, pesquisadores e organizações, sendo capaz de gerar conhecimento, \textit{insights} e valor, servindo como base para responder perguntas e sustentar tomadas de decisão.

    O mundo altamente digitalizado em que vivemos possui dados em abundância, e gera diariamente um grande volume de novas informações. Contudo, grande parte destes dados não está propriamente preparado para análise, seja por questões funcionais, como, por exemplo, dados de bancos transacionais ou então por estarem "sujos", contendo ruído, informações errôneas, formatações confusas e diversos outros problemas de qualidade, tornando longo o caminho entre a coleta de dados e a geração de conhecimento.

    Tendo em vista isso, o presente projeto visa o desenvolvimento de um repositório unificado com dados provenientes de fontes públicas como, por exemplo, IBGE e PNADC, passando pelas etapas de Extract Transform Load (ETL) e/ou Extract Load Transform (ELT), higienização, estruturação e documentação, de modo a disponibilizar uma interface com dados preparados para análise, de forma a encurtar o caminho necessário para o usuário entre a informação e o conhecimento, facilitando assim o acesso e o estudo deles.

\section{O formato dos dados do IBGE}

    Os dados do Instituto Brasileiro de Geografia e Estatística (IBGE) são disponibilizados em dois formatos: os microdados e os agregados, que se diferenciam apenas quanto à sua granularidade.

    Os conjuntos de dados nomeados pelo IBGE de agregados, como diz o nome, são os microdados das pesquisas do instituto agrupados de acordo com certos critérios. A \textit{Application Programming Interface} (API) de serviço de dados do IBGE \cite{API-IBGE} disponibiliza tais dados agregados por localidade, indo desde grande região (p. ex. Norte, Sul, \textit{etc.}) até o grão de município ou distrito municipal (quando existente), por período de tempo, cujo grão é dependente da periodicidade da pesquisa e também por variável disponível na pesquisa.
    
    Já os microdados ``consistem no menor nível de desagregação dos dados de uma pesquisa, retratando, sob a forma de códigos numéricos, o conteúdo dos questionários, preservado o sigilo estatístico com vistas à não individualização das informações.'' \cite{microdados}, ou seja, cada entrada representa o conjunto de respostas de um entrevistado em uma determinada pesquisa. 
    
    Em termos geográficos, enquanto os agregados chegam apenas até o grão de município, os microdados estão também separados por \textbf{setores censitários}, que são definidos por \textcite{Guia-Censo-2010} no Guia do Censo como ``unidades territoriais estabelecidas para fins de controle cadastral, situadas em um único quadro urbano ou rural, com dimensão e número de domicílios [...]''. 

    % EDITAR. talvez mover o final deste parágrafo para outra parte.
    Sendo assim, os microdados são capazes de ser muito mais específicos, contudo com a desvantagem de serem um volume de dados muito maior, o que dificulta seu processamento. Por exemplo, no Censo Demográfico de 2022 foram recenseados mais de 452 mil setores censitários inseridos em 5.568 municípios e os distritos federal e de Fernando de Noronha, e dentro destes setores foram coletadas informações referentes a 75 milhões de domicílios \cite{Guia-Censo-2022}.

% ----------------------------------------------------------
% Finaliza a parte no bookmark do PDF
% para que se inicie o bookmark na raiz
% e adiciona espaço de parte no Sumário
% ----------------------------------------------------------
\phantompart

\chapter{Metodologia}

    Por conta das diferentes formas em que os dados se encontram disponibilizados pelo IBGE discutidas anteriormente, e também pelo caráter dos mesmos quanto à granularidade e quais dados são disponibilizados, foram tomadas duas diferentes abordagens: a de coletar os dados através da API de serviço de dados e a de baixar os arquivos de microdados dos Censos de 2000 e 2010 e "traduzi-los".

\section{APIs do serviço de dados do IBGE}

    O IBGE disponibiliza em seu serviço de dados\footnote{Disponível em <\url{https://servicodados.ibge.gov.br/api/docs/}>. Acessado em 30 de set. de 2023.} ao todo 17 APIs, incluindo a de Agregados, já discutida anteriormente, a de Localidades, que inclui os códigos dos vários níveis de localidade (p. ex. unidade federativa (UF), macrorregião, município), a de Metadados, incluindo informações como periodicidade e variáveis disponíveis em cada uma das pesquisas disponíveis na API, entre outras. Nestas APIs é possível, através de \textit{Uniform Resource Locators} (URLs) gerar requisições por meio de bibliotecas como a \textit{requests} do \textit{python} para obter os dados da API no formato \textit{JavaScript Object Notation} (JSON). Por exemplo, se fizermos uma requisição a API de localidades utilizando a URL \url{https://servicodados.ibge.gov.br/api/v1/localidades/distritos} irá gerar uma lista de valores no formato do exemplo \textit{Listing} \ref{lst:exemplo-api-localidades}:

    \bigskip
    
\begin{lstlisting}[float = h, label={lst:exemplo-api-localidades},language=JavaScript, caption=Exemplo de resultado de uma requisição da API de localidades.]
{"id":421400310, "nome":"Mirador",
  "municipio":{"id":4214003,"nome":"Presidente Getúlio",
    "microrregiao":{"id":42011,"nome":"Rio do Sul",
      "mesorregiao":{"id":4204,"nome":"Vale do Itajaí",
        "UF":{"id":42, "sigla":"SC","nome":"Santa Catarina",
          "regiao":{"id":4, "sigla":"S", "nome":"Sul"}}}},
    "regiao-imediata":{"id":420023, "nome":"Ibirama - Presidente Getúlio",
      "regiao-intermediaria":{"id":4207, "nome":"Blumenau",
          "regiao":{"id":4, "sigla":"S","nome":"Sul"}}}}
}
\end{lstlisting}


% {"id":420690010, "nome":"Dalbérgia",
%   "municipio":{"id":4206900, "nome":"Ibirama",
%     "microrregiao":{"id":42011, "nome":"Rio do Sul", "mesorregiao":{"id":4204, "nome":"Vale do Itajaí",
%         "UF":{"id":42, "sigla":"SC", "nome":"Santa Catarina",
%           "regiao":{"id":4, "sigla":"S", "nome":"Sul"}}}},
%     "regiao-imediata":{"id":420023, "nome":"Ibirama - Presidente Getúlio",
%       "regiao-intermediaria":{"id":4207, "nome":"Blumenau",
%         "UF":{"id":42, "sigla":"SC", "nome":"Santa Catarina",
%           "regiao":{"id":4, "sigla":"S", "nome":"Sul"}}}}}}

    Contudo a formatação JSON, por mais que seja mais eficiente de um ponto de vista transacional, não é ideal para o uso analítico por possuir como característica certa complexidade de navegação e maior custo computacional para criação de consultas, desta forma sendo melhor o armazenamento dos dados em formato tabular, já que persistência e normalização não são preocupações primárias quando refere-se a dados analíticos. Para tal, foi realizado o processo conhecido como \textit{unnesting} (desaninhamento em português), onde os dados de um certo nível de aninhamento são trazidos recursivamente para seu nível superior até que estejam todos em um único nível. O algoritmo em \textit{listing} \ref{lst:unnesting} demonstra como é realizado o processo de \textit{unnesting}:

\bigskip

\begin{lstlisting}[float = h, label={lst:unnesting},language=Python, caption=Algoritmo de \textit{unnesting} em \textit{python}.]
def unnest_json(json_dict: dict, col_name: str = '') -> dict:
    output = {}
    # Define uma função recursiva que "achata" (flatten) o JSON
    def flatten(column, col_name:str = ''):
        # Checa se a coluna é do tipo dict e continua a recursão se sim
        if type(column) == dict:
            # Itera sobre as chaves do dicionário
            for key in column:
                # Determina o nome da coluna para a chave (key) atual
                if col_name == '':
                    new_col_name = str(key)
                else:
                    # adiciona o nome do pai da coluna ao nome da coluna
                    parent_col = col_name.split('_')[-1]
                    new_col_name = parent_col + '_' + str(key)
                # Chama recursivamente a função flatten para a chave (key) atual
                flatten(column[key], new_col_name)
        else:
            # Se a coluna não é do tipo dict, salva o valor dela no output
            output[col_name] = column
    # Chama a função de flatten usando o JSON como parâmetro
    flatten(json_dict, col_name)
    return output
\end{lstlisting}

    Após o \textit{unnesting} feito, obtemos os dados todos em um mesmo nível, sendo assim capazes facilmente de transformá-los em uma tabela como a exemplificada pela tabela \ref{tab:exemplo-api-localidades}:

\begin{center}
    \begin{table}[h]
        \begin{tabular}{c l c l c l}
            \hline
                id-distrito & nome-distrito & id-municipio & nome-municipio & $\dotsi$ & nome-regiao\\
            \hline
                421400310 & Mirador & 4214003 & Presidente Getúlio & $\dotsi$ & Sul\\
                420690010 & Dalbérgia & 4206900 & Ibirama & $\dotsi$ & Sul\\     
            \hline
        \end{tabular}
        \caption{Exemplo dos dados de localidade em formato tabular. Fonte: Dados do autor (2023).}
        \label{tab:exemplo-api-localidades}
    \end{table}
\end{center}

    Um certo problema que pode ainda ocorrer é que, mesmo após o \textit{unnesting}, que os dados em JSON podem conter também elementos em formato de lista, gerando assim uma coluna na tabela de resultado que possui em si uma lista de valores. Portanto se faz necessário um segundo tratamento para estes casos, que é facilmente resolvido armazenando o conjunto de dados dentro de um objeto do tipo \textit{DataFrame} da biblioteca \textit{pandas} da linguagem \textit{python} e utilizar o método \lstinline{pandas.DataFrame().explode()}\footnote{Documentação disponível em <\url{https://pandas.pydata.org/docs/reference/api/pandas.DataFrame.explode.html}>, código fonte disponível em <\url{https://github.com/pandas-dev/pandas/blob/v2.1.1/pandas/core/frame.py\#L9432-L9558}>. Acessado em 30 de set. 2023.}     de modo a converter elementos \textit{list-like} que possam existir dentro das colunas em linhas, copiando os demais valores, como é exemplificado pela figura \ref{fig:exploding-df}.

\begin{figure}[h]
    \centering
    \includegraphics[width=\textwidth]{files/img/exploding_table.png}    \caption{Exemplo do uso do método \lstinline{pandas.DataFrame().explode()}. Fonte: Dados do autor (2023).}
    \label{fig:exploding-df}
\end{figure}

    Nota-se que os valores "explodidos" continuam aninhados, portanto nesse caso foi reaplicada a função \lstinline{unnest_json()} de modo a separar adequadamente os dados.

    [Adicionar: como fiz a carga de localidades, como fiz a carga dos países, procedimento para criar a URL adequada para a carga dos agregados]



\section{Microdados das pesquisas do IBGE}

    Os dados fornecidos pela API de agregados consistem essencialmente na consolidação das respostas individuais de cada um dos entrevistados durante o processo de recenseamento de acordo com critérios locacionais, com seu menor grau de agregação sedo município. Tais informações são organizadas em um formato denominado pelo IBGE como microdados e são disponibilizados através do portal de produtos estatísticos da instituição\footnote{Para mais detalhes, consulte <\url{https://www.ibge.gov.br/estatisticas/todos-os-produtos-estatisticas.html}>. Acesso em 03 out. 2023}, estando disponibilizados em diversas pastas compactadas em formato \textit{.zip} que contém os arquivos de texto nos quais se encontram os dados. No entanto, os dados daqueles arquivos não estão imediatamente prontos para a utilização tal qual uma planilha ou um arquivo CSV, mas sim em um formato comprimido, onde os valores categóricos são codificados através de identificadores (IDs) numéricos, enquanto valores que originalmente possuíam casas decimais tem seus pontos flutuantes removidos. Dessa forma, os dados se apresentam conforme ilustrado pela figura \ref{fig:exemplo-microdado}:

\begin{figure}[h]
    \centering
    \includegraphics[width=\textwidth]{files/img/exemplo_microdado.png}
    \caption{Exemplo de microdados: primeiras 10 linhas da pesquisa de Domicílios do Censo de 2010 no estado de Santa Catarina. Fonte: Dados do Autor (2023).}
    \label{fig:exemplo-microdado}
\end{figure}

    Juntamente com os microdados, é fornecido uma planilha no formato \textit{Open Document Spreadsheet} (ODS) ou Excel que descreve o que cada caractere do arquivo significa, relacionando as categorias e identificadores e a formatação referente às casas decimais das variáveis numéricas. 
    
    Tomando como exemplo a amostra da pesquisa de domicílios do Censo Demográfico de 2010, o arquivo de \textit{layout}\footnote{\label{fn-layout-domi}Arquivo /Documentação/Layout/Layout\_microdados\_amostra.xls. Download em: <\url{https://ftp.ibge.gov.br/Censos/Censo_Demografico_2010/Resultados_Gerais_da_Amostra/Microdados/Documentacao.zip}>. Acesso em 03 out. 2023.} (exemplificado na figura \ref{fig:layout-domi}), temos que as duas primeiras posições do arquivo referem-se variável V0001 à UF, cujo valor 42 corresponde ao Estado de Santa Catarina. Outro exemplo significativo é a variável numérica V0010, Peso Amostral, engloba os dígitos da posição 29 até 44, sendo os 3 primeiros (coluna INT) os dígitos anteriores ao ponto decimal, e as 13 subsequentes sendo as casas de precisão após a vírgula (coluna DEC).

\begin{figure}[h]
    \centering
    \includegraphics[width=\textwidth]{files/img/layout amostra domicilios 2010.png}
    \caption{Recorte da planilha de \textit{layout} da pesquisa de domicílios do Censo Demográfico de 2010. Fonte: Dados do Autor (2023).}
    \label{fig:layout-domi}
\end{figure}

    Variáveis padrões contidas nas pesquisas são as referentes à localização geográfica como município, UF e área de ponderação, assim como o peso amostral, que por sua vez é uma medida de representatividade daquela resposta dentro do escopo da pesquisa.

    De modo a preparar os microdados para que estes possam ser utilizados, foi utilizado o \textit{software} Stata (em sua versão 18.0) para associar os IDs e respectivos valores descritos no arquivo de \textit{layout} e então gerar um arquivo de dados que posteriormente poderá ser utilizado como \textit{dataset}. 
    Com comandos do Stata é possível definir a posição em que se encontram as informações de cada variável, sua tipagem, suas \textit{labels} e respectivos pares chave e valor, para então o arquivo de dados ser processado, que é exemplificado pelo \textit{Listing} \ref{lst:do-file-sample} a seguir:

\begin{lstlisting}[float = h, label={lst:do-file-sample},language=Stata, caption=Exemplo de comandos Stata utilizados para "traduzir" os microdados.]
* Define as posições onde se encontram cada dado
quietly infix                         ///
  byte		V1006		53-53		///
  double		V6204		82-84		///
using `"amostra_domicilios_2010_RJ.txt"', clear
* Define a formatação dos dados numéricos
* Neste caso,  o dado deve assumir a formatação 0000.0
format V6204 %04.1f
* Define as labels para os dados
label var V1006		`"SITUAÇÃO DO DOMICÍLIO"'
label var V6204		`"DENSIDADE DE MORADOR / DORMITÓRIO  "'
* Define os pares chave-valor para cada variável
label define V1006_lbl 1 `" Urbana"', add
label define V1006_lbl 2 `" Rural"', add
* Associa os valores da variável V1006 com os pares chave-valor armazenados em V1006_lbl
label values V1006 V1006_lbl
\end{lstlisting}




\chapter{Estudo de Caso}

\chapter{Conclusão}

Os dados do Instituto Brasileiro de Geografia e Estatística são de enorme valor para diversos setores do país, tanto no âmbito público quanto privado. Tais dados podem ser coletados em dois formatos distintos, um em JSON via API de serviço de dados do instituto e o outro em arquivos de texto codificados de acordo com identificadores presentes em um arquivo de \textit{layout}. Em ambos os formatos, após certo processamento e transformação, é possível

Com base na análise das APIs fornecidas do IBGE, conclui-se que 

% Com base na análise das APIs fornecidas pelo IBGE, conclui-se que essas ferramentas oferecem uma gama diversificada de dados que abrangem desde informações geográficas até indicadores socioeconômicos. Com um total de 17 APIs, destacam-se a de Agregados, que compreende uma extensa variedade de dados de pesquisas, e a de Localidades, que disponibiliza códigos e informações sobre diferentes níveis de localidade, como municípios e regiões.

% A utilização dessas APIs é facilitada por meio de \textit{Uniform Resource Locators} (URLs), permitindo a geração de requisições com bibliotecas como \textit{requests} do \textit{Python}. A ferramenta de \textit{Querie Builder} integrada à plataforma contribui para a construção eficiente dessas URLs, simplificando o processo de obtenção de dados no formato \textit{Javascript Object Notation} (JSON).

% No entanto, a complexidade intrínseca à estrutura JSON pode dificultar a análise direta dos dados. Visando a simplicidade na criação de consultas, foi aplicado um processo de \textit{unnesting} à \textit{string} de resultados das requisições, permitindo a normalização dos dados para análises mais eficazes.

% O exemplo prático de \textit{unnesting} foi demonstrado com a API de localidades, onde a transformação dos dados JSON em um formato tabular facilitou a compreensão e análise. Além disso, a abordagem de lidar com elementos em formato de lista foi explorada, utilizando métodos como \lstinline{pandas.DataFrame().explode()}.

% No processo de carga e análise de dados, foram exploradas três APIs específicas: Localidades, Países e Agregados. Cada uma delas apresentou desafios únicos, desde a manipulação de identificadores geográficos até a elaborada extração de indicadores socioeconômicos e agregados.

% Ao finalizar as análises, conclui-se que as APIs do IBGE oferecem uma rica fonte de dados, abrindo oportunidades para estudos detalhados em diversas áreas, como geografia, economia e demografia. A abordagem adotada para lidar com a complexidade dos dados demonstrou ser eficaz, proporcionando uma base sólida para análises mais aprofundadas e insights valiosos.


% \item Estudar diferentes estratégias de coleta de dados censitários do IBGE;
%     \item Carregar os dados do IBGE via API e processá-los de forma a reestruturar eles em formato tabular;
%     \item Codificar um \textit{script} capaz de ler e associar \textit{labels} aos respectivos microdados de modo a gerar um arquivo de dados do \textit{Stata};
%     \item Demonstrar o uso de ambos conjuntos de dados através de um estudo de caso, demonstrando as diferenças de abrangência e utilização dos dois formatos estudados (API e microdados), gerando visualizações e estatísticas básicas.

% -----------------------------------
% ELEMENTOS PÓS-TEXTUAIS
% -----------------------------------
\postextual
% ----------------------------------

%\bibliography{biblio}
\printbibliography

%\glossary

% ----------------------------------------------------------
% Apêndices
% ----------------------------------------------------------

% ---
% Inicia os apêndices
% ---
\begin{apendicesenv}

% % Imprime uma página indicando o início dos apêndices
\partapendices

\setlength{\absparsep}{18pt} 
\begin{apendicesenv}

\chapter{Códigos}
\label{apend-code}

Este apêndice contém os algoritmos, códigos, \textit{scripts} e funções utilizadas para o desenvolvimento do trabalho que são citadas o longo do decorrer da dissertação. O conjunto completo dos arquivos de código se encontra no diretório \verb|/src| do repositório do \textit{Github} <\url{github.com/GDevigili/TCC-IBGE}>.

    \section{Algoritmo de \textit{Unnesting}}
    \label{apend-unnesting}

    O presente algoritmo realiza o processo de \textit{unnesting}, que visa o nivelamento de uma estrutura de dados em nível único, duplicando dados quando necessário. O processo de desaninhamento sobe os dados de um valor ``filho'' (de grau inferior) até o nível de seu valor ``pai'' (de grau imediatamente superior) de forma recursiva. 

\begin{lstlisting}[label={lst:unnesting},language=Python, caption=Algoritmo de \textit{unnesting} em \textit{python}.]
def unnest_json(json_dict: dict, col_name: str = '') -> dict:
    output = {}
    def flatten(column, col_name:str = ''):
        # Se a coluna é do tipo dict, continua a recursão
        if type(column) == dict:
            for key in column:
                # Determina o nome da coluna para cada chave (key)
                if col_name == '':
                    new_col_name = str(key)
                else:
                    # adiciona o pai da coluna ao seu nome
                    parent_col = col_name.split('_')[-1]
                    new_col_name = parent_col + '_' + str(key)
                # Chama recursivamente a função flatten para a chave atual
                flatten(column[key], new_col_name)
        else:
            # Se a coluna não é do tipo dict, salva o valor dela no output
            output[col_name] = column
    flatten(json_dict, col_name)
    return output
\end{lstlisting}


\section{Algoritmo de conversão de \textit{string} JSON para um objeto do tipo \textit{pandas.DataFrame}}

    O \textit{script} abaixo converte um objeto do tipo \textit{string} contendo um código JSON para um objeto do tipo \verb|DataFrame| da biblioteca \verb|pandas|, para tal, utilizando a função de \textit{unnesting}, presente em \textit{listing} \ref{lst:unnesting}. Um exemplo de uso é a conversão do exemplo em \textit{listing} \ref{lst:exemplo-api-localidades} da sessão \ref{metodoslogia-API} para a tabela \ref{tab:exemplo-api-localidades}.

\begin{lstlisting}[label={lst:make-df},language=Python, caption=Algoritmo de transformação de \textit{string} JSON para um objeto do tipo \textit{pandas.DataFrame}.]
import pandas as pd

def make_df(json_dict: dict) -> pd.DataFrame:
    """Return a DataFrame from a json dict.

    Args:
        json_dict (dict): A dict containing the json data.

    Returns:
        pd.DataFrame: A DataFrame with the data.
    """
    # Create a list to store the rows
    rows = []
    # Iterate over each item on the json dict
    for item in json_dict:
        # Unpack the json dict into a single row dict
        row = unpack_json(item)
        # Add the row to the list
        rows.append(row)
    # Create a DataFrame from the rows list
    df = pd.DataFrame(rows)
    return df
\end{lstlisting}

\section{Adaptador HTTP customizado}

    Devido ao erro \verb|UNSAFE_LEGACY_RENEGOTIATION_DISABLED| encontrado ao tentar realizar requisições às APIs do IBGE utilizando sistemas operacionais baseados em \textit{Linux} (foi utilizado Ubuntu 20 nos testes), foi necessário criar um adaptador customizado do \textit{Hypertext Transfer Protocol} (HTTP) utilizando um contexto legado do protocolo \textit{Secure Sockets Layer} (SSL).

\begin{lstlisting}[label={lst:http-adapt},language=Python, caption=Algoritmo de transformação de \textit{string} JSON para um objeto do tipo \textit{pandas.DataFrame}.]
import requests
import urllib3
import ssl

class CustomHttpAdapter (requests.adapters.HTTPAdapter):
    """Transport adapter that allows us to use custom ssl_context."""
    def __init__(self, ssl_context=None, **kwargs):
        self.ssl_context = ssl_context
        super().__init__(**kwargs)

    def init_poolmanager(self, connections, maxsize, block=False):
        self.poolmanager = urllib3.poolmanager.PoolManager(
            num_pools=connections, maxsize=maxsize,
            block=block, ssl_context=self.ssl_context)

def get_request():
    """Return a requests session with a custom SSL context."""
    ctx = ssl.create_default_context(ssl.Purpose.SERVER_AUTH)
    ctx.options |= 0x4  # OP_LEGACY_SERVER_CONNECT
    session = requests.session()
    session.mount('https://', CustomHttpAdapter(ctx))
    return session

def get_request_json(url: str)-> dict:
    """Return a json from a request."""
    response = get_request().get(url)
    return response.json()
\end{lstlisting}

\section{\textit{Script} para a carga de indicadores da API de países}

    O código a seguir realiza requisições para a API de países do serviço de dados do IBGE, retornando um conjunto de dados com 34 indicadores de 193 países. O módulo \verb|utils| importado no \textit{script} está no diretório \verb|src| do repositório do \textit{Github}. As funções utilizadas são \verb|make_df()| e \verb|get_request_json()| (\textit{listings} \ref{lst:make-df} e \ref{lst:http-adapt}, respectivamente).
    
\begin{lstlisting}[label={lst:api-paises},language=Python, caption=\textit{Script} de carga dos indicadores da API de países.]
import pandas as pd
from utils import *

url_paises = "https://servicodados.ibge.gov.br/api/v1/paises/"
df_paises = make_df(get_request_json(url_paises))
df_paises = df_paises.rename(columns={
    'id_M49': 'id-pais'
    ,'id_ISO-3166-1-ALPHA-2': 'sigla-2'
    ,'id_ISO-3166-1-ALPHA-3': 'sigla-3'
})
url_indicadores = "https://servicodados.ibge.gov.br/api/v1/paises/indicadores/"
df_indicadores = make_df(get_request_json(url_indicadores))

country_list = df_paises['sigla'].unique()
indicator_list = df_indicadores['id'].unique()

df_country_indicators = pd.DataFrame()
for country in country_list:
    # making the request
    url_country = f"https://servicodados.ibge.gov.br/api/v1/paises/{country}/indicadores"
    df_country = make_df(get_request_json(url_country))
    # unpacking the lists
    # take the series dict out of the list (there will be always one or zero values on the list)
    df_country = df_country.explode('series')
    # get the country code (e.g. BR) and country name from the series country and concat it into each of the df_country lines
    df_country = pd.concat([df_country, pd.json_normalize(df_country['series'])], axis = 1)
    # drop the column series as it won't be used anymore
    df_country = df_country.drop('series', axis = 1)
    # unpack the values for each date
    df_country = df_country.explode('serie')
    df_country['ano'] = df_country['serie'].apply(lambda x: list(x.keys())[0] if type(x) == dict else None)
    df_country['valor_indicador'] = df_country['serie'].apply(lambda x: list(x.values())[0] if type(x) == dict else None)
    df_country = df_country.drop('serie', axis = 1)
    df_country_indicators = pd.concat([df_country_indicators, df_country], ignore_index=True)
\end{lstlisting}

\newpage


\end{apendicesenv}

\end{apendicesenv}

% ---

% ----------------------------------------------------------
% Anexos
% ----------------------------------------------------------

% \begin{anexosenv}

% \partanexos

% \end{anexosenv}

%---------------------------------------------------------------------
% ÍNDICE REMISSIVO
%---------------------------------------------------------------------
% \phantompart
% \printindex

\end{document}