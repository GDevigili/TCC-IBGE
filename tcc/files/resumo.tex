\setlength{\absparsep}{18pt} 
\begin{resumo}[Resumo]
 %Segundo a  o resumo deve ressaltar o
 %objetivo, o método, os resultados e as %conclusões do documento. A ordem e a extensão
 %destes itens dependem do tipo de resumo (informativo ou indicativo) e do
 %tratamento que cada item recebe no documento original. O resumo deve ser
 %precedido da referência do documento, com exceção do resumo inserido no
 %próprio documento. (\ldots) As palavras-chave devem figurar logo abaixo do
 %resumo, antecedidas da expressão Palavras-chave:, separadas entre si por
 %ponto e finalizadas também por ponto. Deve ser redigido na terceira
 %pessoa do singular e quanto a sua extensão, o resumo deve ter de 150 a 500
 %palavras.

O presente trabalho objetiva a criação de um repositório de dados públicos, disponibilizando \textit{datasets} coletados através de duas diferentes fontes de dados do Instituto Brasileiro de Geografia e Estatística  (IBGE): a API do serviço de dados e os microdados dos Censos Demográficos. Utilizando a API, foram coletados dados referentes às localidades, indicadores socioeconômicos de diversos países, os chamados agregados e seus metadados. O último grupo consiste da consolidação de respostas das pesquisas promovidas pelo instituto de acordo com diversos critérios, variáveis as quais são armazenadas em arquivos de microdados. Estes arquivos contém de forma anonimizada as respostas individuais de cada entrevistado durante a pesquisa, permitindo análises mais detalhadas do que os agregados. Durante o trabalho, foram tratados de dados do censo demográfico brasileiro, contudo os códigos desenvolvidos são genéricos e podem ser utilizados em outras pesquisas do IBGE desde que sigam o mesmo formato  possuam os mesmos dados. Também foi realizado um estudo de caso, no qual foi exemplificado o uso de ambos os tipos de dados e comparando eles através de visualizações.

Palavras-chave: Censo Demográfico. IBGE. Microdados. Dados públicos.

\end{resumo}

\begin{resumo}[Abstract]
 \begin{otherlanguage*}{english}

This work aims to create a public data repository, making available datasets collected from two different sources from Brazilian Institute of Geography and Statistics (IBGE): the data service API and the microdata from the brazilian demographic census. Using the API, data related to locations, socioeconomic indicators from various countries, the so-called aggregations, and their metadata were collected. The last group consists of the consolidation of many variables from surveys made by the institute according to diverse criteria, whose variables are stored in text files called microdata files. These files anonymize individual answers from each of the intervewee that took part of the census and they allow their user to make more detailed data analysis than they could if using the aggregates. Throughout the project, Brazilian demographic census data were processed and treated; however, the codes are generic and can be used in other IBGE surveys as long as they follow the same format and have the same files. Also, there were made data visualizations in order to exemplify and compare both datasets.


 \end{otherlanguage*}

 Keywords: Brazilian demographic census. IBGE. microdata. Public data.
\end{resumo}